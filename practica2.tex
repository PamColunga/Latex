\documentclass[letterpaper,11pt]{article}

\usepackage[utf8]{inputenc}
\usepackage[T1]{fontenc}
\usepackage[spanish]{babel}
\usepackage{multicol}
\usepackage{enumerate}
\usepackage{slantsc}

\topmargin = -2cm
\oddsidemargin= 0cm
\textheight = 23cm
\textwidth = 17cm

\renewcommand{\shorthandsspanish}{}

\parindent=0mm

\title{Mi primer documento en \LaTeX{}}

\author{Pablo Pérez}

\date{\today{}}

\begin{document}

\thispagestyle{empty}

\maketitle

\tableofcontents

\newpage

\begin{abstract}
Ejemplo de documento \LaTeX{} de la clase {\ttfamily article} con una estructura básica. 
Incluye secciones, subsecciones y una referencia cruzada. También se incluyen varios detalles básicos, el uso de diferentes características de las fuentes, párrafos y listas.
\end{abstract}

\section{Primera sección}\label{primera}

\LaTeX{} nos permite crear distintas clases de documentos, entre las que se pueden mencionar: book, article, report, letter y presentaciones, entre otras.

Todos ellos tienen diferentes características y formato. Por ejemplo, la clase \textit{"book"} (libro) coloca un estilo diferente a las páginas pares e impares, permite crear documentos divididos en capítulos, introducir diferente  tipo de tablas de contenido, entre otros detalles.





En \LaTeX los espacios y saltos de linea no siempre se toman de forma literal.
Por ejemplo, un espacio        es      equivalente a      dos o más. 
Un salto de linea es equivalente a un espacio en blanco.

Dos o más saltos de linea son equivalentes a un salto de linea.

Se pueden introducir saltos de linea mediante la expresión $\backslash\backslash$.

Se puede introducir un salto de página mediante el comando $\backslash$newpage o \textbackslash pagebreak.

\section{Caracteres especiales}

\LaTeX hace uso de